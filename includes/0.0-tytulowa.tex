\thispagestyle{empty}

\vspace*{2em}

\begin{center}
    \includegraphics{pjatk}
\end{center}

\vspace{3em}

\begin{center}
    \large
    \textbf{Wydział Informatyki}\\
    Specjalizacja: Technologie sieci urządzeń\\
    mobilnych oraz chmury obliczeniowej

    \vspace{3em}

    \textbf{Szymon Kogut}\\
    Numer albumu: 24271\\

    \vspace{3em}

    \begin{center}
        \LARGE
        \textbf{Porównanie modeli scentralizowanych~i~rozproszonych w~wirtualnych sieciach prywatnych}
        \large\\
        \vspace{1em}
        Comparison of centralized and distributed\\ models in virtual private networks
    \end{center}

    \vspace{3em}

    \hfill
    \begin{varwidth}{\linewidth}
        \raggedright
        \textbf{Rodzaj pracy}\\
        Magisterska\\

        \vspace{1em}

        \textbf{Imię i nazwisko promotora}\\
        dr Tadeusz Puźniakowski
    \end{varwidth}
\end{center}

\vfill

\begin{center}
    Warszawa \today
\end{center}

\newpage
{
    \large
    \textbf{Streszczenie:} Celem pracy jest weryfikacja różnych modeli i protokołów sieci wirtualnych pod kątem stabilności w restrykcyjnym środowisku oraz łatwości utrzymania w projektach o niskim stopniu złożoności infrastrukturalnej.\\
    
    Porównaniem objęto następujące protokoły: OpenVPN (topologia scentralizowana), Nebula (topologia rozproszona) oraz WireGuard (obie topologie). Przygotowano skrypty automatyzujące proces wdrażania.\\
    
    W ramach badań przeprowadzono testy wydajnościowe przepustowości, opóźnień i obciążenia zasobów. Zbadano stabilność połączeń w restrykcyjnych warunkach sieciowych oraz oceniono skalowalność poszczególnych rozwiązań przy zwiększaniu liczby węzłów.\\

    W pracy dodatkowo zawarto ocenę łatwości wdrożenia poszczegolnych rozwiązań.\\\\

    \textbf{Słowa kluczowe:} vpn, openvpn, nebula, wireguard
}