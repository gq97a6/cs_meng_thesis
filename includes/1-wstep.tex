\section{Wstęp}
%Słowa obce i pojęcia specyficzne dla kontekstu oznaczone są \tcbox{szarym tłem}.

Internet początkowo rozwijał się jako zdecentralizowana sieć tworzona oddolnie przez niezależne podmioty. Z czasem usługi świadczone za jego pośrednictwem zyskały na znaczeniu a wraz z tym uległy monopolizacji przez duże korporacje.\\

Aspekty takie jak suwerenność danych, ochrona prywatności, ograniczenie kosztów czy potrzeba autonomii to czynniki motywujące użytkowników indywidualnych oraz małe przedsiębiorstwa do zwrócenia się w stronę samodzielnego utrzymywania infrastruktury usług na potrzeby własne.\\

Niezastąpione przy takim podejściu są sieci wirtualne. Łączą one urządzenia, niezależnie od ich fizycznej lokalizacji. Jest to niezbędne dla zachowania pełni funkcjonalności w porównaniu z komercyjnymi rozwiązaniami.\\

Niniejsza praca ma na celu przegląd dostępnych rozwiązań z zakresu sieci wirtualnych. Celem jest znalezienie optymalnego rozwiązania dostosowanego do skali projektów o niskim stopniu złożoności infrastrukturalnej.\\

W związku z powyższym, przy analizie skupiono się nie tylko na pomiarze syntetycznej wydajności poszczególnych rozwiązań, ale również zbadano skalowalność, łatwość wdrożenia oraz stabilność pracy w restrykcyjnych warunkach sieciowych.