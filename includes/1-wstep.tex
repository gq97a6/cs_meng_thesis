\section{Wstęp}

\subsection{Motywacje}

Internet początkowo rozwijał się jako zdecentralizowana sieć tworzona oddolnie przez niezależne podmioty. Z czasem usługi świadczone za jego pośrednictwem zyskały na znaczeniu a wraz z tym uległy monopolizacji przez duże korporacje.\\

Aspekty takie jak suwerenność danych, ochrona prywatności, ograniczenie kosztów czy potrzeba autonomii to czynniki motywujące użytkowników indywidualnych oraz małe przedsiębiorstwa do zwrócenia się w stronę samodzielnego utrzymywania infrastruktury usług na potrzeby własne.\\

Niezastąpione przy takim podejściu są sieci wirtualne. Łączą one urządzenia, niezależnie od ich fizycznej lokalizacji. Jest to niezbędne dla zachowania pełni funkcjonalności w porównaniu z komercyjnymi rozwiązaniami.

\subsection{Cel}

Sieci nastawione na użytkowników końcowych stanowią wyzwanie dla każdego kto chce zajmować się utrzymaniem usług na własną rękę - brak publicznego adresu IPv4, restrykcyjny wariant NAT-u, brak możliwości administracji routerem brzegowym, niska stabilność łącza.\\

Praca ma na celu przedstawienie czytelnikowi, jakie problemy można napotkać oraz jakie są wady i zalety dostępnych rozwiązań w zależności od priorytetów danego projektu, wraz z opisem procesu ich wdrażania.\\

W związku z powyższym, przy analizie skupiono się nie tylko na pomiarze syntetycznej wydajności poszczególnych rozwiązań, ale między innymi zbadano skalowalność, łatwość wdrożenia oraz stabilność pracy w restrykcyjnych warunkach sieciowych.

\subsection{Prace powiązane}

\cite{rw1}
\cite{rw2}
\cite{rw3}
\cite{rw4}
\cite{rw5}
\cite{rw6}

