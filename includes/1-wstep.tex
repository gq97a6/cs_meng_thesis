\section{Wstęp}

\subsection{Motywacje}

Internet początkowo rozwijał się jako zdecentralizowana sieć tworzona oddolnie przez niezależne podmioty. Z czasem usługi świadczone za jego pośrednictwem zyskały na znaczeniu a wraz z tym uległy monopolizacji przez duże korporacje.\\

Aspekty takie jak suwerenność danych, ochrona prywatności, ograniczenie kosztów czy potrzeba autonomii to czynniki motywujące użytkowników indywidualnych oraz małe przedsiębiorstwa do zwrócenia się w stronę samodzielnego utrzymywania infrastruktury usług na potrzeby własne.\\

Z uwagi na niską adopcję IPv6 oraz ograniczenia nakładane przez dostawców usług internetowych stawiane dla prosumentów zasobów sieciowych, niezastąpione przy takim podejściu są sieci wirtualne. Łączą one urządzenia, niezależnie od ich fizycznej lokalizacji. Jest to niezbędne dla zachowania pełni funkcjonalności w porównaniu z komercyjnymi rozwiązaniami.

\subsection{Cel}

Sieci nastawione na użytkowników końcowych stanowią wyzwanie dla każdego kto chce zajmować się utrzymaniem usług na własną rękę - brak publicznego adresu IPv4, restrykcyjny wariant NAT-u, brak możliwości administracji routerem brzegowym, niska stabilność łącza.\\

Praca ma na celu analizę wad i zalet dostępnych rozwiązań w domenie sieci wirtualnych oraz ocenę procesu ich wdrażania.\\

W związku z powyższym, przy analizie skupiono się nie tylko na pomiarze syntetycznej wydajności, ale między innymi, zbadano również skalowalność, łatwość wdrożenia oraz stabilność pracy w restrykcyjnych warunkach sieciowych.

\subsection{Prace powiązane}

Temat wydajności protokołów sieci VPN jest obiektem wielu prac badawczych. Głównym celem, większości tych prac, jest porównanie syntetycznej wydajności osiąganej przez te protokoły, poprzez pomiary wykonywane w laboratoryjnych warunkach. \cite{rw1,rw2,rw4,rw6}\\

Niniejsza praca różni się w tym zakresie od tych dostępnych, ponieważ kładzie nacisk na rozwiązania o niskim progu wejścia, zakładając, że użytkownik końcowy operuje w nietypowym środowisku sieciowym bez dostępu do infrastruktury klasy korporacyjnej czy zaawansowanego wsparcia technicznego.