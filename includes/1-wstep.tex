\section{Wstęp}
%Słowa obce i pojęcia specyficzne dla kontekstu oznaczone są \tcbox{szarym tłem}.

\section{Architektury i protokoły sieci wirtualnych}
\subsection{Modele architektury}
\subsection{Problematyka łączności w sieciach IP}
\subsection{Charakterystyka badanych protokołów}

\section{Metodyka badań}
\subsection{Narzędzia pomiarowe i metryki} %Opisujesz iperf3, ping, sar/htop. Definiujesz co mierzysz (Delay, Jitter, Throughput, CPU Load).
\subsection{Metodyka oceny złożoności wdrożenia} %Tu opisujesz swoje kryteria "łatwości" (czas wdrożenia, liczba linii configu).
\subsection{Scenariusze testowe}

\section{Projekt i implementacja}
\subsection{Automatyzacja procesu wdrażania}
\subsection{Konfiguracja OpenVPN}
\subsection{Konfiguracja WireGuard w modelu rozproszonym}
\subsection{Konfiguracja WireGuard w modelu scentralizowanym}
\subsection{Problemy napotkane podczas implementacji}

\section{Analiza wyników}
\subsection{Badanie wydajności sieciowej}
\subsection{Analiza obciążenia zasobów systemowych}
\subsection{Odporność na trudne warunki sieciowe}
\subsection{Analiza skalowalności}
\subsection{Ocena złożoności konfiguracji i utrzymania}

\section{Podsumowanie}
\subsection{Synteza wyników}
\subsection{Wnioski końcowe}