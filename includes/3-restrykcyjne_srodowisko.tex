\section{Restrykcyjne środowisko sieciowe}
Celem pracy jest znalezienie rozwiązania możliwego do wdrożenia w sieciach konsumenckich. Poniższy rozdział definiuje ograniczenia związane z tym środowiskiem.

\subsection{Rodzaje NAT}
Jest to decydujący czynnik utrudniający nawiązywanie połączeń z urządzeniami spoza sieci lokalnej. W zależności od typu \tcbox{NAT} za jakimi znajdują się urządzenia, będzie to proces utrudniony bądź w pełni uniemożliwiony.

\subsubsection{Static NAT}
Przydziela publiczny adres każdemu urządzeniu z puli publicznych adresów. Z tego powodu nie stanowi bariery przy nawiązywaniu bezpośrednich połączeń między urządzeniami. Niewykorzystywany dla rozwiązań konsumenckich.

\subsubsection{Dynamic NAT}
Rozwiązanie rzadziej stosowane działające podobne do statycznego. Jedyna różnica jest taka, że adresy przydzielane są w sposób dynamiczny.

\subsubsection{Network Address and Port Translation (NAPT)}
Zastosowanie \tcbox{Port Address Translation (PAT)} umożliwia współdzielenie jednego publicznego adresu IP przez wiele urządzeń.\\

Zasada działania tego mechanizmu opiera się na użyciu portów dla rozróżnienia poszczególnych klientów. Każdy port, każdego urządzenia w sieci wewnętrznej, otrzymuje na żądanie port zewnętrzny routera. \tcbox{NAPT} działa jak warstwa translacji. Dla zapytań wychodzących podmienia źródłowy adres IP oraz port na własne. Dla zapytań przychodzących podmienia docelowy adres oraz port na te należące do odbiorcy wewnątrz sieci.\\

\begin{figure}[h]
    \centering
    \includegraphics[width=1\textwidth]{nat_diagram}
    \caption{Schemat działania mechanizmu NAT/PAT}
    \label{fig:natdiagram}
\end{figure}

\newpage

Poszczególne implementacje \tcbox{NAPT} różnią się poziomem restrykcyjności odnośnie filtrowania przychodzących pakietów.\\

\textbf{Full Cone NAT}\\
Najbardziej permisywna implementacja. Nie wprowadza dodatkowych ograniczeń. Mapowanie portów może być predefiniowane lub definiowane w odpowiedzi na wysyłane zapytania.\\

\textbf{Restricted Cone NAT}\\
W tej implementacji host wewnętrzny musi najpierw wysłać zapytanie kierowane na dany adres. Zakres akceptowanych pakietów jest zawężony do pakietów z tym adresem źródłowym.\\

\textbf{Port Restricted Cone NAT}\\
W tej implementacji host wewnętrzny musi najpierw wysłać zapytanie kierowane na dany adres oraz port. Zakres akceptowanych pakietów jest zawężony do pakietów z tym adresem źródłowym oraz portem.\\

\textbf{Symmetric NAT}\\
Najbardziej restrykcyjna implementacja. Posiada ograniczenia adresu i port z tą różnicą że każda ich kombinacja otrzymuje dedykowany port zewnętrzny. Wcześniejsze implementacje współdzielą wybrany port zewnętrzny pomiędzy adresami docelowymi.\\

W przypadku gdy oba urządzenia znajdują się za tego typu \tcbox{NAT}-em, nie ma możliwości nawiązania bezpośredniego połączenia pomiędzy nimi.

\subsubsection{Carrier-grade NAT}
Jest to specyficzny przypadek zastosowania \tcbox{NAPT}. Odnosi się on do jego wykorzystania w sieci wewnętrznej dostawcy internetu. Użytkownik współdzieli jeden adres z innymi klientami dostawcy. Istnieje wiele implementacji \tcbox{CGNAT}. Natomiast główny podział wynika z wykorzystywanego protokołu.\\

Popularny mechanizm \tcbox{DS-Lite} zakłada że klient nie otrzymuje żadnego adresu IPv4 a jedynie IPv6. Pakiety IPv4 są enkapsulowane w pakiety IPv6 przy przesyle do routera brzegowego dostawcy.\\

Wart ponownego odnotowania jest fakt, że jest to specyficzny sposób wykorzystania \tcbox{NAPT}, aniżeli jego konkretna implementacja. Przykładowo, CGNAT może być w implementacji \tcbox{Symmetric}, ale równie dobrze może to być \tcbox{Full Cone NAT}.

\newpage

\subsection{Wysokie opóźnienia i jitter}
VPN z definicji nadaje dodatkowe opóźnienie ponad to bazowe, wynikające z sieci użytkownika. W modelu rozproszonym, opóźnienia są zminimalizowane ponieważ poszczególne urządzenia komunikują się bezpośrednio ze sobą. W modelu scentralizowanym, każdy pakiet musi niejako nadrabiać drogę poprzez przejście przez serwer zanim trafi do odbiorcy.\\

Jest to zależne od zastosowania, jednak opóźnienie może mieć na tyle duży wpływ by kategorycznie wykluczyć wykorzystanie sieci wirtualnych w pewnych przypadkach.

\subsection{Utrata pakietów}
W tunelach UDP, utrata pakietów powoduje braki w danych, ale nie zatrzymuje całej transmisji. Natomiast w tunelach TCP może ona spowodować zjawisko TCP Meltdown. Zarówno protokół VPN, jak i aplikacja wewnątrz tunelu próbują jednocześnie retransmitować zgubione pakiety. Skutkuje to lawinowym wzrostem opóźnień, drastycznym spadkiem przepustowości i częstym zrywaniem sesji.

\subsection{Blokada protokołu UDP}
Blokuje działanie najwydajniejszych protokołów, które domyślnie korzystają z UDP.
