\section{Restrykcyjne środowisko sieciowe}
Celem pracy jest znalezienie rozwiązania możliwego do wdrożenia w sieciach konsumenckich. Poniższy rozdział definiuje ograniczenia związane z tym środowiskiem.

\subsection{Rodzaje NAT}
Jest to decydujący czynnik utrudniający nawiązywanie połączeń z urządzeniami spoza sieci lokalnej. W zależności od typu \tcbox{NAT} za jakimi znajdują się urządzenia, będzie to proces utrudniony bądź w pełni uniemożliwiony.

\subsubsection{Static NAT}
Przydziela publiczny adres każdemu urządzeniu z puli publicznych adresów. Z tego powodu nie stanowi bariery przy nawiązywaniu bezpośrednich połączeń między urządzeniami. Nie wykorzystywany dla rozwiązań konsumenckich.

\subsubsection{Dynamic NAT}
Rozwiązanie rzadziej stosowane działające podobne do statycznego. Jedyna różnica jest taka, że adresy przydzielane są w sposób dynamiczny.

\subsubsection{Network Address and Port Translation (NAPT)}
Zastosowanie \tcbox{Port Address Translation (PAT)} umożliwia współdzielenie jednego publicznego adresu IP przez wiele urządzeń.\\

Zasada działania tego mechanizmu opiera się na użyciu portów dla rozróżnienia poszczególnych klientów.\\

\textit{Flow diagram}\\

%Zastąpić diagramem
%\begin{enumerate}[leftmargin=*]
%    \item \textbf{Host wewnętrzny wysyła zapytanie do zewnętrznego hosta:}\\
%    $HostWewnętrznyIP_{src}:Port_{src}$
%    Od \tcbox{PC:3000} do \tcbox{SERVER:80}.
%    
%    \item \textbf{Router wybiera losowy wolny port zewnętrzny:}\\
%    Port \tcbox{5000} lub port równy oryginalnemu w przypadku braku konfliktu.
%    
%    \item \textbf{Router zapamiętuje mapowanie:}\\
%    Zewnętrzny port \tcbox{5000} jest przeznaczony dla \tcbox{PC:3000}.
%    
%    \item \textbf{Router podmienia IP oraz port docelowy w zapytaniu wychodzącym:}\\
%    Z \tcbox{PC:3000} na \tcbox{ROUTER:5000}.
%    
%    \item \textbf{Host zewnetrzny odbiera zapytanie:}\\
%    Od \tcbox{ROUTER:5000} do \tcbox{SERVER:80}.
%    
%    \item \textbf{Host zewnętrzny odsyła odpowiedź:}\\
%    Od \tcbox{SERVER:80} do \tcbox{ROUTER:5000}.
%    
%    \item \textbf{Router podmienia IP oraz port docelowy w zapytaniu przychodzącym:}\\
%    Z \tcbox{ROUTER:5000} na \tcbox{PC:3000}.
%    
%    \item \textbf{Host wewnętrzny odbiera odpowiedź:}\\
%    Od \tcbox{SERVER:80} do \tcbox{PC:3000}.
%\end{enumerate}

\newpage

Poszczególne implementacje \tcbox{NAPT} różnią się poziomem restrykcyjności odnośnie filtrowania przychodzących pakietów.\\

\textbf{Full Cone NAT}\\
Najbardziej permisywna implementacja. Nie wprowadza dodatkowych ograniczeń. Mapowanie portów może być predefiniowane lub definiowane w odpowiedzi na wysyłane zapytania.\\

\textbf{Restricted Cone NAT}\\
W tej implementacji host wewnętrzny musi najpierw wysłać zapytanie kierowane na dany adres. Zakres akceptowanych pakietów jest zawężony do pakietów z tym adresem źródłowym.\\

\textbf{Port Restricted Cone NAT}\\
W tej implementacji host wewnętrzny musi najpierw wysłać zapytanie kierowane na dany adres oraz port. Zakres akceptowanych pakietów jest zawężony do pakietów z tym adresem źródłowym oraz portem.\\

\textbf{Symmetric NAT}\\
Najbardziej restrykcyjna implementacja. Posiada ograniczenia adresu i port z tą różnicą że każda ich kombinacja otrzymuje dedykowany port zewnętrzny. Wcześniejsze implementacje współdzielą wybrany port zewnętrzny pomiędzy adresami docelowymi.

\subsubsection{Carrier-grade NAT}
Jest to specyficzny przypadek zastosowania \tcbox{NAPT}. Odnosi się on do jego wykorzystania w sieci wewnętrznej dostawcy internetu. Użytkownik współdzieli jeden adres z innymi klientami dostawcy. Istnieje wiele implementacji \tcbox{CGNAT}. Natomiast główny podział wynika z wykorzystywanego protokołu.\\

Popularny mechanizm \tcbox{DS-Lite} zakłada że klient nie otrzymuje żadnego adresu IPv4 a jedynie IPv6. Pakiety IPv4 są enkapsulowane w pakiety IPv6 przy przesyle do routera brzegowego dostawcy.\\

Wart ponownego odnotowania jest fakt, że jest to specyficzny sposób wykorzystania \tcbox{NAPT}, aniżeli jego konkretna implementacja. Przykładowo, CGNAT może być w implementacji \tcbox{Symmetric}, ale równie dobrze może to być \tcbox{Full Cone NAT}.

\subsection{Wysokie opóźnienia i Jitter}
\subsection{Utrata pakietów}
\subsection{Blokada protokołu UDP}

