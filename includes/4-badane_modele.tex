\section{Badane modele}
\section{Model scentralizowany}

Cały ruch sieciowy jest przekierowywany przez centralny serwer. Nawet gdy oba urządzenia znajdują się blisko siebie, przesył danych dalej odbywa się z wykorzystaniem serwera jako pośrednika.\\

Konfiguracja i weryfikacja tożsamości użytkowników następuje na serwerze. Umożliwia to zarządzanie całą siecią, bez konieczności ponownego wdrażania poszczególnych klientów.\\

Sieć w oparciu o publiczny serwer, jako hub dla wszystkich węzłów, usuwa potrzebę nawiązywania bezpośrednich połączeń. Jest to duże uproszczenie w przypadku gdy klient znajduje się za symetrycznym \tcbox{NAT}-em.\\

Centralny węzeł oznacza również pojedynczy punkt awarii oraz wąskie gardło. Przepustowość sieci jest ograniczona wydajnością łącza i procesora serwera centralnego.

\section{Model rozproszony}

Połączenia nawiązywane są bezpośrednio pomiędzy klientami sieci. Zmniejsza to opóźnienia do minimum. Połączenie może być nawiązane bez użycia dodatkowych serwerów. Opcjonalny serwer służy jedynie do wymiany informacji w procesie inicjalizacji połączenia dla klientów za \tcbox{NAT}-em.\\

Obciążenie rozkłada się na poszczególne węzły, nie ma więc wąskiego gardła. Znika również pojedynczy punkt awarii, zwiększając odporność sieci.\\

Dla pewnych zastosowań, brak centralizacji stacje się minusem. Włączanie nowych klientów do sieci jest utrudnione. Nie ma centralnej kontroli nad działaniem poszczególnych węzłów.\\

